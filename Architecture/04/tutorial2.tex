\documentclass[12pt, a4paper]{article}

\usepackage{graphicx}
\graphicspath{ {/home/zakariyya/Downloads/} }

\usepackage{blindtext}
\usepackage{geometry}
\geometry{
  a4paper,
  total={170mm,257mm},
  left=20mm,
  top=20mm,
}

\begin{document}
\section{Question 1}

\begin{center}
  \begin{tabular} {|c|c|c|}
    \hline
    \textbf{Input} & \textbf{Output} & \textbf{Storage} \\
    \hline
    Keyboard & Monitor & SSD \\
    Mouse & Speaker & HDD \\
    Touch Screen & Printer & Micro SD \\
    Controller & Projector & RAM \\
    Microphone & Speech Synthesizer & ROM \\
    \hline
  \end{tabular}
\end{center}

\section{Question 2}
A Direct Memory Access (DMA) controller allows devices to send and receive
data from the main memory while allowing the CPU to do other tasks. Through
the memory bus, the CPU tells the controller the length of the memory chunk
it is supposed to copy, the device and memory address, the command (read
or write). While the CPU performs other tasks, the DMA controller performs
required operations. The CPU and the DMA controller are in competition for 
the memory bus but since the CPU will find most of its data in the cache,
it will leave cycles for the DMA controller to use the bus.

\section{Question 2}
A device driver is a special software that is written for hardware devices.
They allow software to communicate with hardware. More specifically, they
allow the OS and applications to work with hardware. They also allow 
different pieces of hardware to work together. Drivers are needed because
different pieces of hardware may use different protocols for I/O. Drivers
can abstract away all of the hardware's details into a streamlined API which
the OS and applications may use. There are different kinds of drivers: 
\begin{itemize}
  \item Kernel device drivers - Drivers for generic devices such as 
    motherboard and BIOS.
  \item Original Equiment Manufacturer(OEM) drivers - proprietary drivers
    installed seperately after installing the OS.
  \item Virtual device drivers - used to control virtual machines and 
    emulate hardware of the host device.
\end{itemize}

\section{Question 4}
True. Updating drivers can fix issues with certain software.
For instance, newer applications may
be using the latest drivers to communicate with hardware. Sticking to an
older version may lead to compatibility issues. In other cases, driver updates
leads to performance gains. For examples, the latest drivers for the 
Intel ARC GPUs yielded a 119 \% performance improvement in DX11 and DX12 games.

\section{Question 5}
A KVM switch's functions is to control and switch between different PCs
using a single trifecta of keyboard, monitor and switch. It is useful
when a user has different PCs for different tasks. For instance, streamers
who have their main PC and a secondary PC for streaming.

\section{Question 6}
\begin{itemize}
  \item Refresh rate - the rate at which the screen is redrawn. 
  \item Pixel Pitch - the vertical and horizontal distance between individual pixels.
  \item Resolution - the number of pixels in the display
  \item Viewing angle - the maximum angular distance from looking in front of a screen
    to looking from the side without a change in brightness or colors.
\end{itemize}

\section{Question 7}
100ms

\section{Question 8}

FAT32 is the 32 bit ver

\end{document}
