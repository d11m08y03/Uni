\documentclass[12pt, a4paper]{article}

\usepackage{graphicx}
\graphicspath{ {/home/zakariyya/Downloads/} }

\usepackage{blindtext}
\usepackage{geometry}
\geometry{
  a4paper,
  total={170mm,257mm},
  left=20mm,
  top=20mm,
}

\usepackage{amsmath}
\usepackage{amssymb}
\setlength{\parindent}{0pt}

\newcommand{\bbb}{\paragraph{}\mbox{}\\}
\newcommand{\ex}{\; \exists \;}
\newcommand{\real}{\mathbb{R}}
\newcommand{\nat}{\mathbb{N}}
\newcommand{\all}{\; \forall \;}

\newcommand{\joe}{2\sqrt{2}}
\newcommand{\brom}{\sqrt{3+2m}}
\newcommand{\bron}{\sqrt{3+2n}}
\newcommand{\rom}{\sqrt{4+3m}}
\newcommand{\ron}{\sqrt{4+3n}}

\title{Tutorial 1}

\begin{document}
\maketitle

\section{Question 1}
a. There exists at least one subsequence of $a_n$ that does not converge.\\
b. $\exists \; x_n \geq 5 \ni x_n \notin S $.\\
c. $x \notin A \cup B $.\\
c. $x \notin A \cap B $.\\
d. $\exists \; x \in (a, b) \ni f(x) < 0$.\\
e. For all $K \in \real, \ex x \in [-a, a] \ni |f(x) \leq k|$

\section{Question 2}
\begin{gather*}
\text{Prove that} \lim_{ n \to \infty } \frac{n}{n+1} = 1.
\text{ Given } \epsilon > 0, \text{by the Archimedean property of } \real,
  \ex K \in \nat  \ni \\ \frac{1}{K+1} < \epsilon 
  \implies \frac{1}{n + 1} \leq
  \frac{1}{K+1} < \epsilon \all n \geq K 
  \implies \bigg|\frac{n}{n+1} - 1\bigg| = \bigg|\frac{n}{n+1} - \frac{n+1}{n+1}\bigg|
  \\[5pt]= \bigg |\frac{1}{n+1} \bigg| = 
  \frac{1}{n+1} < \epsilon \all n \geq K.
\end{gather*}

\begin{gather*}
  \text{Prove that} \lim_{ n \to \infty } \frac{n}{n+2 \sqrt{2} } = 1.
  \text{ Given } \epsilon > 0, \text{by the Archimedean property of } \real,
  \ex K \in \nat \ni \\[5pt] \frac{\joe}{K + \joe} < \epsilon \implies 
  \frac{\joe}{n + \joe} \leq \frac{\joe}{K + \joe} < \epsilon \all n \geq K \implies
  \bigg| \frac{n}{n + \joe} - 1 \bigg| = \bigg| \frac{\joe}{n + \joe} \bigg| =\\[5pt] 
  \frac{\joe}{n + \joe} < \epsilon \all n \geq K.
\end{gather*} 

\section{Question 3}
\includegraphics[scale=0.5]{kappa}

\section{Question 4}
\includegraphics[scale=0.5]{kappa}

\section{Question 5}
Since $(a_n)$ is a null sequence, $\lim_{n \to \infty} a_n = 0 \implies
$ given $ \epsilon > 0, \ex K \in \nat \ni |a_n - 0| = |a_n| < \epsilon 
\all n \geq K \implies K |a_n| < K \cdot \epsilon \implies 
|b_n - l| < K|a_n| < K \cdot \epsilon. $ Yep death.

\section{Question 6}
\includegraphics[scale=0.5]{kappa}

\section{Question 7 (Sus)}

\subsection{Part 1}
\begin{gather*}
  \text{Given } \epsilon > 0, \text{ choose }
  N \in \nat \ni N \geq \frac{\epsilon^2}{8} - \frac{3}{2}. \text{ Then,}
  \all m, n > N,\\[5pt] 
  | \brom - \bron | = | -\bron + \brom | \leq \bron + \brom < \\[5pt]
  \sqrt{3 + 2 \cdot (\frac{\epsilon^2}{8} - \frac{3}{2})} 
  + \sqrt{3 + 2 \cdot (\frac{\epsilon^2}{8} - \frac{3}{2})} =
  \sqrt{3 + \frac{\epsilon^2}{4} - 3} + \sqrt{3 + \frac{\epsilon^2}{4} - 3} = \epsilon \\[5pt]
  \implies a_n \text{ is a Cauchy sequence } \implies a_n \text{ is bounded.}
\end{gather*}

\subsection{Part 2}
\begin{gather*}
  \text{Let } f(x) = \sqrt{3 + 2x}. \; \; f'(x) = \frac{1}{\sqrt{3 + 2x}}.
  \text{ Since } a_0 = 0, \;\; \sqrt{3 + 2x} \text{ will always be positive.} \\[5pt]
  \implies f'(x) \text{ will always be positive } \implies f(x) \text{ is an increasing function. } \\[5pt]
  \implies a_n \text{ is a monotone increasing sequence by comparison.}
\end{gather*}

Since $a_n$ is monotone and bounded, it is convergent according to the
monotone convergence theorem.

\section{Question 8 (Sus)}

\subsection{Part 1}
\begin{gather*}
  \text{Given } \epsilon > 0, \text{ choose }
  N \in \nat \ni N \geq \frac{\epsilon^2}{12} - \frac{4}{3}. \text{ Then,}
  \all m, n > N,\\[5pt] 
  | \rom - \ron | = | -\ron + \rom | \leq \ron + \rom < \\[5pt]
  \sqrt{4 + 3 \cdot (\frac{\epsilon^2}{12} - \frac{4}{3})} 
  + \sqrt{4 + 3 \cdot (\frac{\epsilon^2}{12} - \frac{4}{3})} =
  \sqrt{4 + \frac{\epsilon^2}{4} - 4} + \sqrt{4 + \frac{\epsilon^2}{4} - 4} = \epsilon \\[5pt]
  \implies a_n \text{ is a Cauchy sequence } \implies a_n \text{ is bounded.}
\end{gather*}

\subsection{Part 2}
\begin{gather*}
  \text{Let } f(x) = \sqrt{4 + 3x}. \; \; f'(x) = \frac{1}{\sqrt{4 + 3x}}.
  \text{ Since } a_0 = 6, \;\; \sqrt{4 + 3x} \text{ will always be positive.} \\[5pt]
  \implies f'(x) \text{ will always be positive } \implies f(x) \text{ is an increasing function. } \\[5pt]
  \implies a_n \text{ is a monotone increasing sequence by comparison.}
\end{gather*}

\end{document}
