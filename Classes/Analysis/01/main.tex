\documentclass[12pt, a4paper]{article}

\usepackage{amsmath}
\usepackage{amssymb}
\setlength{\parindent}{0pt}

\newcommand{\bbb}{\paragraph{}\mbox{}\\}
\newcommand{\ex}{\; \exists \;}
\newcommand{\real}{\mathbb{R}}
\newcommand{\nat}{\mathbb{N}}
\newcommand{\all}{\; \forall \;}

\title{El Recap De Comer}

\begin{document}
\maketitle

\section{Real Numbers}

\subsection{Completeness}
Completeness is a property of $\mathbb{R}$ that implies that there are no
"gaps" in the real number line. It states that every non-empty set of 
$\real$ that is bounded has a supremum and an infimum in $\real$. 

\subsection{Archimedean Property of $\real$}
It states that $\forall \; \epsilon > 0, \ex n \in \nat \ni \frac{1}{n}
< \epsilon$.
This means that there are no infinitesimally small elements in the real
line, no matter how small $\epsilon$ gets, we will always be able to find
an even smaller positive real number in the form $\frac{1}{n}$.

\subsection{Boundedness}
A set of real numbers is said to be \textbf{bounded} if it it bounded
above and below. A subset $S$ of $\real$ is said to be bounded $\iff
|s| < \alpha \all s \in S \implies -\alpha < s < \alpha \all s \in S$.

\subsection{Infimum and Supremum}
Let $S$ be a subset of $\mathbb{R}$. $S$ is said to be \textbf{bounded above}
if there is a real number $\alpha \ni \alpha \geq s \; \forall \; s \in S$.
Such a number $\alpha$ is called an \textbf{upperbound} of $S$.
$S$ is \textbf{bounded below} if $\exists \; \beta \in \mathbb{R} \ni \beta
\leq s \; \forall \; s \in S$. $\beta$ is called a \textbf{lowerbound} of
set $S$. \\

$\alpha$ is the \textbf{least upperbound / supremum} of $S \iff$:
\begin{itemize}
  \item $\alpha \geq s \; \forall \; s \in S$
  \item given $\epsilon > 0 \ex s \in S \ni \alpha - \epsilon < s$ 
\end{itemize}

$\beta$ is the \textbf{greatest upperbound / infinum} of $S \Longleftrightarrow$:
\begin{itemize}
  \item $\beta \leq s \; \forall \; s \in S$
  \item given $\epsilon > 0 \ex s \in S \ni \beta + \epsilon > s$ 
\end{itemize}

\section{Sequences}

\subsection{Limits And Convergence}
Consider an infinite sequence $(a_n) = a_1, a_2, a_3, ...$ of real numbers.
% If there is a real number $l$ with the property that, given any degree of
% closeness to $l$ there is a point in the sequence after which all terms
% are within that degree of closeness, we say that the sequence \textbf{converges}
% to the \textbf{limit} $l$. Thus, $ \lim_{n\to\infty} (a_n) = l$
The sequence is said to converge if $ \exists \; L \in \real \ni $ for any given
$ \epsilon > 0 \ex K \in \real \ni |x_n - L| < \epsilon \all n \geq K$. In
this case, we say that $(x_n)$ has limit $L$ and we write:
\begin{gather*}
  \lim_{n \to \infty} x_n = L
\end{gather*}

\subsection{Example}
Using the definition of the limit of a sequence, prove that 
$\lim_{n \to \infty} (1/n)= 0$. \\

\textbf{Proof:} Given $e > 0$, by the Archimedean property of $\real,
\ex K \in \nat \ni 1/K < \epsilon.$ Then, if $n \geq K$, then $1/n \leq
1/K < \epsilon \implies |x_n - 0| = |1/n - 0| = 1/n \leq 1/K < \epsilon$.
This proves, by definition, that $\lim_{n \to \infty} \frac{1}{n} = 0$.

\subsection{Propositions}
\begin{itemize}
  \item Every convergent sequence is bounded. Thus, if $(a_n)$ is convergent
    then $\exists \; \alpha \in \real \ni |a_n| < \alpha \all n$.

  \item Suppose $\lim_{n \to \infty} a_n = a$ and $\lim_{n \to \infty}
    b_n = b$. Then:
    \begin{gather*}
      \lim_{n \to \infty} (a_n + b_n) = a + b \\[2pt]
      \all k \in \real, \lim_{n \to \infty} (k \cdot a_n) = k \cdot a \\[2pt]
      \lim_{n \to \infty} (a_n \cdot b_n) = ab \\[2pt]
      \text{if } b \ne 0, \lim_{n \to \infty} \frac{a_n}{b_n} = \frac{a}{b}[2pt]
    \end{gather*}

  \item Let $(a_n), (b_n)$ and $(c_n)$ be sequences satisfying
    $(a_n) \leq (b_n) \leq (c_n) \all n > k$, where k is some fixed 
    positive integer. If $(a_n)$ and $(c_n)$ both converge to the same
    limit $l$, then $(b_n)$ also converges to $l$.

  \item A sequence $(a_n)$ is said to \textbf{tend to infinity} [$(a_n) \to \infty$]
    if given $K \in \real, \ex N \in \nat \ni n \geq N \implies a_n > K$.

  \item The sequence $(a_n)$ is said to \textbf{tend to minus infinity} 
    [$(a_n) \to -\infty$]
    if given $K \in \real, \ex N \in \nat \ni n \geq N \implies a_n < K$.
\end{itemize}

\subsection{Monotonic Sequences}
Let $(x_n)$ be a sequence of real number. We say that:
\begin{itemize}
  \item $(x_n)$ is monotonic increasing if $x_{n+1} \geq x_n \all n \in \nat$.
  \item $(x_n)$ is monotonic strictly increasing if 
    $x_{n+1} > x_n \all n \in \nat$.
  \item $(x_n)$ is monotonic decreasing if $x_{n+1} \leq x_n \all n \in \nat$.
  \item $(x_n)$ is monotonic strictly decreasing if $x_{n+1} < x_n \all n \in \nat$.
\end{itemize}

\subsection{Monotone convergence theorem}
If $(x_n)$ is bounded and monotone then:
\begin{itemize}
  \item if it is bounded above and increasing: 
    \begin{gather*}
      \lim_{n \to \infty} x_n = Sup(x_n)
  \end{gather*}
  \item if it is bounded below and decreasing: 
    \begin{gather*}
      \lim_{n \to \infty} x_n = Inf(x_n)
    \end{gather*}
\end{itemize}

\textbf{Proof:} Suppose that $(x_n)$ is bounded above and increasing. 
By the completeness property of $\real$, $Sup(x_n)$ exists. Let $\alpha = Sup(x_n)$
and $\epsilon > 0$ be arbitrary. Then, by the properties of the supremum,
$\exists \; x_K \ni \alpha - \epsilon < x_K \leq \alpha$. Since $(x_n)$
is increasing and $\alpha$ is an upperbound for the range of the sequence,
it follows that $x_K \leq x_n \leq \alpha \all n \geq K \implies
\alpha - \epsilon < x_n < \alpha + \epsilon \all n \geq K \implies 
-\epsilon < x_n - \alpha < \epsilon \implies |x_n - \alpha| < \epsilon
\all n \geq K$.\\

If $(x_n)$ is not bounded and monotone then:
\begin{itemize}
  \item if it is increasing, then $(x_n) \to \infty$.
  \item if it is decreasing, then $(x_n) \to -\infty$.
\end{itemize}

\subsection{Subsequences}
Let $(x_n)$ be a sequence. A subsequence of $(x_n)$ is a sequence of the
form $(x_{n_1}, x_{n_2}, x_{n_3}, ...)$ where $n_1 < n_2 < n_3 < ...$ is
a sequence of strictly increasing natural numbers. A subsequence of $(x_n)$
will be denoted by $(x_{n_k})$.

\subsection{Theorems}
Let $(x_n)$ be a sequence.
\begin{itemize}
  \item if $(x_n) \to L$ then $(x_{n_k}) \to L$.
  \item if $(x_n)$ has two subsequences converging to distinct limits,
    $(x_n)$ is divergent.
  \item if $(x_n)$ has a subsequence that diverges, then $(x_n)$ is divergent.
  \item if $(x_n)$ is bounded, then $(x_{n_k})$ is also bounded.
  \item if $(x_n)$ is monotonic increasing, then $(x_{n_k})$ is also monotonic
    increasing.
\end{itemize}

\subsection{Bolzano - Weierstrass theorem}
It states that every bounded sequence has a convergent subsequence.

\end{document}
