\documentclass{article}

\usepackage{amsmath}
\usepackage{amssymb}
\usepackage{tikz}
\usepackage{mathtools}
\usepackage{pgfplots}
\setlength{\parindent}{0pt}

\newcommand\Mycomb[2][^n]{\prescript{#1\mkern-0.5mu}{}C_{#2}}

\begin{document}

\section{Negative Binomial Distribution (NBD)}
A Bernoulli trial is an experiment that can result is either a
'success' or a 'failure', but not both. \\

Consider a sequence of Bernoulli trials with probability of success
$ p $ and probability of failure $ q $ such that $ 0 \leq p \leq 1 $
and $ p + q = 1 $. If $ X $ is the number of failures before the
$ r^{th} $ success, X is said to follow a NBD with parameters $ r $
and $ p $, denoted by:

\begin{gather*}
  X \sim \text{NBin}(r, p)
\end{gather*}

\subsection{Probability Mass Function Of NBD}

\begin{gather*}
  P(X = n) = {{n + r - 1} \choose {r - 1}} p^r q^n, 
  \text{ for } n \in \mathbb{N}, \text{ where } q = 1-p
\end{gather*}

\subsection{Expected Value and Variance of NBD}
\begin{gather*}
  E(X) = \frac{r(1-p)}{p} \\[5pt]
  Var(X) = \frac{r(1-p)}{p^2}
\end{gather*}

\subsection{Conditons for NBD}
\begin{itemize}
  \item Experiment must 2 mutually exclusive outcomes denoted as
    'success' or 'failure'
  \item Probability of success must be constant for each trial
  \item Each trial must be independent
  \item The experiment must have a finite number of success(es)
\end{itemize}

\subsection{Relationship between Binomial Distribution and NBD}
Consider $ n $ independent Bernoulli trials with the same probability 
of success $ p $. If $ Y $ is the number of successes, it is said
to follow a binomial distribution with parameters $ n $ and $ p $.

\begin{gather*}
  Y \sim Bin(n, p)
\end{gather*}

Upon comparison, both the BD and NBD are based upon independent 
Bernoulli trials. However, they differ in what they are counting.
The BD counts the number of succeses in a fixed number of 
trials $ n $ while the NBD counts the number of failures until
a fixed number of succeses $ r $.

\subsection{Illustration of NBD}
To illustrate, we will use data from Statistics Mauritius pertaining
to grades of student in Economics A Level during the 2023 seating.
Below is a summary of the data collected, along with mean and 
variance.

\begin{center}
  \begin{tabular}{|c|c|c|c|c|c|c|}
    \hline
    \textbf{Grade} & \textbf{Point Range} & \textbf{$f_i$} & \textbf{$x_i$} & \textbf{$ f_i \cdot x_i $} & \textbf{$ (x_i - \mu)^2 $} & \textbf{$ f_i \cdot (x_i - \mu)^2  $}\\
    \hline
    \hline
    A* & 129-180	& 75	& 154.5	& 11587.5	& 5731.441922	& 429858.1441 \\
    A  & 113-129	& 261	& 120.5	& 31450.5	& 1739.414392	& 453987.1564 \\
    B  & 95-112	& 435	& 103.5	& 45022.5	& 610.4006273	& 265524.2729 \\
    C  & 83-95	  & 419	& 89	  & 37291	  & 104.1682985	& 43646.51705 \\
    D  & 71-83	  & 513	& 77	  & 39501	  & 3.2174056	  & 1650.529073 \\
    E  & 60-71	  & 490	& 65.5	& 32095	  & 176.7227999	& 86594.17197 \\
    F  & 0-60	  & 495	& 30	  & 14850	  & 2380.826409	& 1178509.072 \\
    \hline
  \end{tabular}
\end{center}

\begin{gather*}
  \mu = \frac{\sum {f_i \cdot x_i}}{\sum f_i} = 78.79 \\[5pt]
  \sigma = \frac{1}{\sum f_i} \cdot \sum {f_i \cdot (x_i - \mu)^2} = 915.09
\end{gather*}

\textbf{Scenario:} Consider an event
where A-Level economics students are gathered, the event organiser
wants an $ r^{th} $ number of students who obtained A* and starts 
approaching attendees about their grades. Let the random variable
$ X $ represent the probability that the organiser gathers
those students after $ n $ number of attempts. From this, the
following NBD can be constructed:

\begin{gather*}
  X \sim \text{NBD}(3, 0.0279)
\end{gather*}

Taking a sample of 50, the rnbinom function in R outputs the following:

\begin{center}
  \begin{tabular}{|c|c|c|c|c|c|c|c|c|c|c|c|c|}
    \hline
    x & 0 & 1 & 2 & 3 & 4 & 5 & 6 & 7 & 8 & 9 & 10 & 11 \\
    \hline
    f & 4 & 10 & 8 & 6 & 8 & 6 & 3 & 4 & 0 & 0 & 0 & 1\\
    \hline
  \end{tabular}
\end{center}

Based on the above, the mean and variance can be calculated:
\begin{gather*}
  \mu = \frac{\sum x \cdot f}{\sum f} = \frac{163}{50} = 3.26 \\[5pt]
  \sigma^2 = \frac{(x_i - \mu)^2}{n - 1} = 5.46
\end{gather*}

\subsubsection{Bar Chart}

\subsubsection{Issues with this application}

\section{Hypergeometric Distribution (HD)}
Consider a population of $N$ objects which are divided into 2 types: type A and type B.
There are
$n$ objects of type $A$ and $N$ - $n$ objects of type $B$.
Suppose a random sample of size $r$ is taken
(without replacement) from the entire population of $N$ objects. If $X$ 
is the number of objects of
type $A$ in the sample,then $X$ follows a HD with parameters $n$, 
$N-n$ and r denoted by:

\begin{gather*}
  X \sim \text{HGeom}(n, N-n, r)
\end{gather*}

\subsection{Probability Mass Function of HD}

\begin{gather*}
  p(k) =  \frac{\Mycomb[n]{k} \cdot \Mycomb[(N-n)]{(r-k)}}
  {\Mycomb[N]{r}}, \; \;
  \text{for } max\{ 0, r - (N - n) \} \leq k \leq min\{r, n\}
\end{gather*}

\subsection{Expected Value and Variance of HD}
\begin{gather*}
  E(X) = \frac{nr}{N} \\[5pt]
  Var(X) = \frac{nr}{N} \cdot \frac{N -r}{N} \cdot \frac{N-n}{N-1}
\end{gather*}

\subsection{Conditions for HD}
\begin{itemize}
  \item Finite population
  \item Population can be seperated into 2 types
  \item Sampling is done without replacement (dependent trials).
\end{itemize}

\end{document}
