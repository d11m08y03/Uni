\documentclass[12pt, a4paper]{article}

\usepackage{amsmath}
\usepackage{amssymb}
\setlength{\parindent}{0pt}
\usepackage{geometry}
\geometry{
  a4paper,
  total={170mm,257mm},
  left=20mm,
  top=20mm,
}

\newcommand{\bbb}{\paragraph{}\mbox{}\\}
\newcommand{\ex}{\; \exists \;}
\newcommand{\real}{\mathbb{R}}
\newcommand{\nat}{\mathbb{N}}
\newcommand{\all}{\; \forall \;}
\newcommand{\lp}{\left(}
\newcommand{\rp}{\right)}
\newcommand{\san}{\sum_{n = 1}^{\infty} a_n}
\newcommand{\sbn}{\sum_{n = 1}^{\infty} b_n}

\begin{document}

\section{Series}

Given an infinite series:
\begin{gather*}
  \sum_{k = 1}^{\infty} a_k
\end{gather*}

\begin{itemize}
  \item the numbers $a_k$ are the terms of that series

  \item the sequence of partial sums is the sequence: 
    \begin{gather*}
      (S_n) = \lp \sum_{k = 1}^{n} a_k \rp_{n = 1}^{\infty}
      = a_1 + a_2 + a_3 + ... + a_n \\[5pt]
      \text{The series converges to } L \in \real, \text{ written }
      \sum_{k = 1}^{\infty} a_k = L, \text{ if } (S_n) \rightarrow L,
      \text{and diverges} \\[5pt]
      \text{(to $ \infty, -\infty $ or}
      \text { does not exist) if $(S_n)$ does. Likewise, we say the
      series is bounded or} \\[5pt]
      \text{monotone if $S_n$ is.} 
    \end{gather*}
\end{itemize}

\section{Geometric Series}
A geometric series can be written in the form:
\begin{gather*}
  \sum_{n = 1}^{\infty} a \cdot r^n = a + ar + ar^2 + ...
\end{gather*}
Its partial sum, $S_n$ can be written as:
\begin{gather*}
  S_n = \frac{a(1 - r^n)}{1 - r} \implies
  \lim_{n \to \infty} S_n = \frac{a}{1 - r}
\end{gather*}

A geometric series will always converge if $|r| < 1$.

% TODO: Telescopic Series %

\section{Harmonic Series}
A harmonic series can be written in the form:
\begin{gather*}
  \sum_{k = 1}^{\infty} \frac{1}{K} = 1 + \frac{1}{2} + \frac{1}{3} + ...
\end{gather*}
This series diverges.

\section{P-Series}
If p is a real constant, a p-series can be written in the form:
\begin{gather*}
  \sum_{n = 1}^{\infty} \frac{1}{n^p} = \frac{1}{1^p} + \frac{1}{2^p} + ...
\end{gather*}

A p-series converges if $p > 1$ and diverges if $p \leq 1$.

\section{The N-th Term Test}
\begin{flalign*}
  & \text{If} \lim_{n \to \infty} a_n \neq 0
  \text{ then } \sum_{n = 1}^{\infty} a_n \text{ diverges. }
  \text{As such, if } \sum_{n = 1}^{\infty} a_n \text{ converges, }
  \lim_{n \to \infty} a_n = 0
\end{flalign*}

This test only guarantees that a series diverges if the terms do not
go to 0 in the limit. It cannot prove convergence.

\section{Comparison Test For Convergence}
\begin{gather*}
  \text{Given } \sum_{n = 1}^{\infty} a_n \text{ and }
  \sum_{n = 1}^{\infty} b_n \text{. Then if } a_n, \; b_n \geq 0, \;
  a_n \leq b_n \all n \text{ and } \sum_{n = 1}^{\infty} b_n 
  \text{ converges, then } \\[5pt]
  \sum_{n = 1}^{\infty} a_n \text{ converges. Similarly, if }
  \sum_{n = 1}^{\infty} b_n  \text{ diverges, then }
  \sum_{n = 1}^{\infty} a_n \text{ diverges.}
\end{gather*}

\section{Limit Comparison Test}
\begin{gather*}
  \text{Given } \san \text{ and } \sbn, \text{ then, if } a_n,
  \; b_n > 0 \all n > 0 \text{ and } \lim_{n \to \infty} \frac{a_n}{b_n}
  \text{ is positive and finite, then } \\[5pt]
  \san \text{ and } \sbn \text{ are either both convergent or divergent.}
\end{gather*}

\section{The Ratio Test}
\begin{gather*}
  \text{Given } \san. \text{ If } \lim_{n \to \infty} 
  \frac{a_{n + 1}}{a_n} = L, \text{ then the series}: \\[5pt] 
  i. \text{ converges if } L < 1 \\[5pt]
  ii. \text{ diverges if } L > 1 \\[5pt]
  iii. \text{ may or may not converge if } L = 1
\end{gather*}

\end{document}
